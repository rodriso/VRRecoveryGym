\apendice{Plan de Proyecto Software}

\section{Introducción}

\section{Planificación temporal}
La planificación temporal de este proyecto, debido a que se esta siguiendo la metodologia Scrum, sera realizada mediante \textit{sprints}.

\subsection{\textbf{\textit{sprint} 1: Planificación del proyecto y estructura del juego}}
\begin{itemize}
\item {Planning meeting}:
Durante la reunión se marcaron los siguientes objetivos:

\begin{enumerate}
\item \textbf{Revisión de documentación de puzles:}
  Leer y analizar toda la documentación relacionada con los puzles para comprender los requisitos,  
  mecánicas y desafíos que deberán implementarse.

\item \textbf{Planificación de la estructura del juego:}
  Definir la arquitectura general del juego, estableciendo los módulos  
  principales, el flujo de navegación y la integración de los puzles.

\item \textbf{Selección de gafas de realidad virtual:}
  Acordar la versión de las gafas de \textit{RV} a utilizar y definir la  
  preferencia basada en la compatibilidad, rendimiento y experiencia de usuario.
\end{enumerate}

    \item {Duración del sprint:}
El \textit{sprint } se realizo entre el 12 de marzo de 2025 y el 17 de marzo de 2025, con una duración de 7 días.

    \item {Burndown Report}
Este \textit{sprint} se cumplió con casi todos los objetivos en el plazo determinado. La selección de las gafas de realidad virtual se veía retrasada debido a que la persona que iba a prestarlas para este proyecto no se encontraba disponible es ese momento. Cabe destacar que durante la duración de este \textit{sprint} también se realizaron tareas de aprendizaje sobre Unreal Engine ya que los conocimientos eran nulos sobre este motor gráfico. 

    \item {Sprint review meeting}
Durante esta reunión, se resolvieron  dudas sobre el sistema de registro que debía tener la aplicación, como se querían estructurar las partidas y el requisito de toma de algunos datos que pedía la organización ASPAYM que eran inviables en el nivel de este proyecto. Los \textit{issues} fueron creados de forma correcta, pero se marco que había que planificar los \textit{sprints} correspondientes al proyecto.
\end{itemize}
\subsection{\textbf{Sprint 2: Menu principal } }
\begin{itemize}
    \item {Planning meeting:}
    
    Durante la reunión se marcaron los siguientes objetivos:
    \begin{enumerate}
        \item \textbf{Registro de usuarios de forma local:} El registro se realizará mediante el \textit{savegame} de la partida, donde, si es la primera vez que se inicia el juego en el dispositivo, solo se pedirá al usuario el nombre. De esta manera, se asegura que el registro sea rápido y sencillo. 
    
        \item \textbf{Creación de nuevas partidas:} Cada partida será asociada a un nivel, el cual se seleccionará como el puzle específico que se quiere jugar. Esto implica que cada partida estará vinculada a un puzle que servirá como contenido principal de la misma. 
    
        \item \textbf{Listado de las partidas creadas:} La aplicación deberá permitir visualizar una lista de las partidas y proporcionar información básica sobre cada puzle, como el nombre del puzle, el nivel y el estado del progreso (si corresponde). 
    \end{enumerate}
    
    
        \item Duración del sprint:
    
    El \textit{sprint } se realizo entre el 26 de marzo de 2025 y el 9 de abril de 2025, con una duración de 14 días.
    
    
        \item {Burndown Report:}
    
    Este \textit{sprint} se cumplió con los objetivos en el plazo determinado. La estimación de tiempo fue muy extensa debido a que se utilizo este plazo para seguir adquiriendo conocimientos sobre \textit{Unreal Engine} mientras se realizaban los objetivos del sprint.
    
    
    
    \item {Sprint review meeting}
    
    Se habló sobre la importancia de agrupar los elementos de la interfaz en \textit{plantillas} como los botones e \textit{items} para la reusabilidad, ya que no tuve en cuenta algunos elementos por lo cual ese aspecto se planteó cambiar para el siguiente \textit{sprint}.
\end{itemize}

\subsection{\textbf{Sprint 3: Plantilla de puzle e integración de menu principal a VR } }
\begin{itemize}
    \item {Planning meeting:}
    
Durante la reunión se marcaron los siguientes objetivos:
\begin{enumerate}
  \item \textbf{Integración de interfaces a realidad virtual:}  
    Adaptar y conectar todas las pantallas y controles del juego al entorno RV, garantizando la navegación y la interacción fluida.

  \item \textbf{Contador de tiempo y almacenamiento de nivel:}  
    Implementar un temporizador que comience al iniciar el puzzle y guarde el tiempo transcurrido junto con los datos del nivel creado en la sesión.

  \item \textbf{Cargado de una partida:}  
    Desarrollar la funcionalidad para seleccionar y restaurar el estado de una partida previamente guardada, incluyendo el progreso y el nivel activo.

  \item \textbf{Visualización de resultados del puzzle:}  
    Mostrar al finalizar cada puzle los siguientes datos:  
    \begin{itemize}
      \item Tiempo empleado  
      \item Fecha de creación de la sesión  
      \item Nombre del puzle jugado
      \item Completado o no completado
    \end{itemize}
\end{enumerate}
    \item {Duración del \textit{sprint}}
    
El \textit{sprint } se realizo entre el 9 de abril de 2025 y el 23 de abril de 2025, con una duración de 14 días.
    \item {Burndown Report}
    
Este \textit{sprint} no se cumplió con los objetivos en el plazo determinado. La estimación de tiempo puede que fuera la correcta pero la falta del\textit{hardware} retrasa las pruebas del software debido a no tener acceso a unas gafas de realidad virtual por la festividad de semana santa.Se focalizo más en la parte de documentación del proyecto.

\subsection{\textbf{Sprint 4: Palacio de Shahriar: Preparación e importación inicial } }
\begin{itemize}
    \item {Planning meeting:}
    
Durante la reunión se marcaron los siguientes objetivos:
\begin{enumerate}
  \item \textbf{Importación de modelos STL:} Importación de todos los modelos 3D del puzzle, con su correcta escala y colisiones. 

  \item \textbf{Arreglar el menú de pausa:} Reestructuración del menú de pausa, debido a que era poco practica la implementación realizada.  

  \item \textbf{Implementar lógica de las torres:}  Realizar una manera de hacer posible realizar giros de rosca en las torres con diferentes mecánicas como laberinto, giro escalonado o rotación acumulada.

  \item \textbf{Añadir botón de saltar cinemática}

\end{enumerate}
    \item {Duración del \textit{sprint}:}
    
El \textit{sprint} se realizo entre el 23 de abril de 2025 y el 7 de mayo de 2025, con una duración de 14 días.
    \item {Burndown Report}
    
Este \textit{sprint} no se cumplió con los objetivos en el plazo determinado. La estimación de tiempo puede que fuera la correcta de nuevo, pero una vez más la falta del \textit{hardware} retrasa las pruebas del software debido a no tener acceso a unas gafas de realidad virtual por compatibilidad de horarios con los prestantes de las gafas de RV. Debido a que comencé las practicas y el horario establecido para que pudiese utilizar las gafas fue muy reducido, tome la decisión de pedirlas prestadas a un amigo para tener total disponibilidad de ellas y que el proyecto no se viese más atrasado por falta de material para las pruebas del código.  
\end{itemize}
\section{Estudio de viabilidad}

\subsection{Viabilidad económica}

\subsection{Viabilidad legal}


