\apendice{Plan de Proyecto Software}

\section{Introducción}
El presente anexo recoge los aspectos fundamentales del plan de desarrollo seguido durante la realización del proyecto. En él se detalla la planificación temporal adoptada, basada en la metodología ágil \textit{Scrum}, así como un análisis de la viabilidad económica y legal del sistema desarrollado.

La planificación con \textit{Scrum} ha permitido dividir el trabajo en iteraciones controladas, fomentar la adaptación continua a los cambios propuestos por el cliente, y mantener una visión clara del avance del proyecto a lo largo del tiempo. Se presentan los \textit{sprints} realizados, el \textit{product backlog}, y los artefactos generados durante las distintas fases de desarrollo.

Por otro lado, se incluye un estudio de la viabilidad económica, donde se estiman los costes asociados al desarrollo, licencias, recursos y herramientas utilizadas. También se expone la viabilidad legal, analizando las licencias de software empleadas y definiendo un marco legal que permita el uso, distribución o ampliación futura del sistema respetando los derechos de terceros.

Este anexo sirve como complemento técnico y organizativo a la memoria principal, aportando una visión estructurada de la gestión del proyecto desde una perspectiva profesional.

\section{Planificación temporal}
La planificación temporal de este proyecto, debido a que se esta siguiendo la metodologia Scrum, sera realizada mediante \textit{sprints}.

\subsection{\textbf{\textit{Sprint} 1: Planificación del proyecto y estructura del juego}}
\begin{itemize}
\item {Planning meeting}:
Durante la reunión se marcaron los siguientes objetivos:

\begin{enumerate}
\item \textbf{Revisión de documentación de puzles:}
  Leer y analizar toda la documentación relacionada con los puzles para comprender los requisitos,  
  mecánicas y desafíos que deberán implementarse.

\item \textbf{Planificación de la estructura del juego:}
  Definir la arquitectura general del juego, estableciendo los módulos  
  principales, el flujo de navegación y la integración de los puzles.

\item \textbf{Selección de gafas de realidad virtual:}
  Acordar la versión de las gafas de \textit{VR} a utilizar y definir la  
  preferencia basada en la compatibilidad, rendimiento y experiencia de usuario.
\end{enumerate}

    \item {Duración del sprint:}
El \textit{sprint} se realizó entre el 12 de marzo de 2025 y el 17 de marzo de 2025, con una duración de 7 días.

    \item {Burndown Report:}
En este \textit{sprint} se cumplió con casi todos los objetivos en el plazo determinado. La selección de las gafas de realidad virtual se veía retrasada debido a que la persona que iba a prestarlas para este proyecto no se encontraba disponible es ese momento. Cabe destacar que durante la duración de este \textit{sprint} también se realizaron tareas de aprendizaje sobre Unreal Engine ya que los conocimientos eran nulos sobre este motor gráfico. 

    \item {Sprint review meeting}
Durante esta reunión, se resolvieron  dudas sobre el sistema de registro que debía tener la aplicación, como se querían estructurar las partidas y el requisito de toma de algunos datos que pedía la organización ASPAYM que eran inviables en el nivel de este proyecto. Los \textit{issues} fueron creados de forma correcta, pero se marco que había que planificar los \textit{sprints} correspondientes al proyecto.
\end{itemize}
\subsection{\textbf{Sprint 2: Menú principal } }
\begin{itemize}
    \item {Planning meeting:}
    
    Durante la reunión se marcaron los siguientes objetivos:
    \begin{enumerate}
        \item \textbf{Registro de usuarios de forma local:} El registro se realizará mediante el \textit{savegame} de la partida, donde, si es la primera vez que se inicia el juego en el dispositivo, solo se pedirá al usuario el nombre. De esta manera, se asegura que el registro sea rápido y sencillo. 
    
        \item \textbf{Creación de nuevas partidas:} Cada partida será asociada a un nivel, el cual se seleccionará como el puzle específico que se quiere jugar. Esto implica que cada partida estará vinculada a un puzle que servirá como contenido principal de la misma. 
    
        \item \textbf{Listado de las partidas creadas:} La aplicación deberá permitir visualizar una lista de las partidas y proporcionar información básica sobre cada puzle, como el nombre del puzle, el nivel y el estado del progreso (si corresponde). 
    \end{enumerate}
    
    
        \item Duración del sprint:
    
    El \textit{sprint } se realizó entre el 26 de marzo de 2025 y el 9 de abril de 2025, con una duración de 14 días.
    
    
        \item {Burndown Report:}
    
    En este \textit{sprint} se cumplió con los objetivos en el plazo determinado. La estimación de tiempo fue muy extensa debido a que se utilizo este plazo para seguir adquiriendo conocimientos sobre \textit{Unreal Engine} mientras se realizaban los objetivos del sprint.
    
    
    
    \item {Sprint review meeting}
    
    Se habló sobre la importancia de agrupar los elementos de la interfaz en \textit{plantillas} como los botones e \textit{items} para la reusabilidad, ya que no tuve en cuenta algunos elementos por lo cual ese aspecto se planteó cambiar para el siguiente \textit{sprint}.
\end{itemize}

\subsection{\textbf{Sprint 3: Plantilla de puzle e integración de menú principal a VR } }
\begin{itemize}
    \item {Planning meeting:}
    
Durante la reunión se marcaron los siguientes objetivos:
\begin{enumerate}
  \item \textbf{Integración de interfaces a realidad virtual:}  
    Adaptar y conectar todas las pantallas y controles del juego al entorno VR, garantizando la navegación y la interacción fluida.

  \item \textbf{Contador de tiempo y almacenamiento de nivel:}  
    Implementar un temporizador que comience al iniciar el puzle y guarde el tiempo transcurrido junto con los datos del nivel creado en la sesión.

  \item \textbf{Cargado de una partida:}  
    Desarrollar la funcionalidad para seleccionar y restaurar el estado de una partida previamente guardada, incluyendo el progreso y el nivel activo.

  \item \textbf{Visualización de resultados del puzle:}  
    Mostrar al finalizar cada puzle los siguientes datos:  
    \begin{itemize}
      \item Tiempo empleado  
      \item Fecha de creación de la sesión  
      \item Nombre del puzle jugado
      \item Completado o no completado
    \end{itemize}
\end{enumerate}
    \item {Duración del \textit{sprint}}
    
El \textit{sprint } se realizó entre el 9 de abril de 2025 y el 23 de abril de 2025, con una duración de 14 días.
    \item {Burndown Report}
    
En este \textit{sprint} no se cumplió con los objetivos en el plazo determinado. La estimación de tiempo puede que fuera la correcta pero la falta del\textit{hardware} retrasa las pruebas del software debido a no tener acceso a unas gafas de realidad virtual por la festividad de semana santa.Se focalizo más en la parte de documentación del proyecto.

\end{itemize}

\subsection{\textbf{Sprint 4: Palacio de Shahriar: Preparación e importación inicial } }
\begin{itemize}
    \item {Planning meeting:}
    
Durante la reunión se marcaron los siguientes objetivos:
\begin{enumerate}
  \item \textbf{Importación de modelos STL:} Importación de todos los modelos 3D del puzle, con su correcta escala y colisiones. 

  \item \textbf{Arreglar el menú de pausa:} Reestructuración del menú de pausa, debido a que era poco practica la implementación realizada.  

  \item \textbf{Implementar lógica de las torres:}  Realizar una manera de hacer posible realizar giros de rosca en las torres con diferentes mecánicas como laberinto, giro escalonado o rotación acumulada.

  \item \textbf{Añadir botón de saltar cinemática}

\end{enumerate}
    \item {Duración del \textit{sprint}:}
    
El \textit{sprint} se realizó entre el 23 de abril de 2025 y el 7 de mayo de 2025, con una duración de 14 días.
    \item {Burndown Report}
    
En este \textit{sprint} no se cumplió con los objetivos en el plazo determinado. La estimación de tiempo puede que fuera la correcta de nuevo, pero una vez más la falta del \textit{hardware} retrasa las pruebas del software debido a no tener acceso a unas gafas de realidad virtual por compatibilidad de horarios con los prestantes de las gafas de VR. Debido a que comencé las practicas y el horario establecido para que pudiese utilizar las gafas fue muy reducido, tome la decisión de pedirlas prestadas a un amigo para tener total disponibilidad de ellas y que el proyecto no se viese más atrasado por falta de material para las pruebas del código. 

\end{itemize}

\subsection{\textbf{\textit{Sprint} 5: Palacio de Shahriar: Interacciones avanzadas }}
\begin{itemize}
\item {Planning meeting}:
Durante la reunión se marcaron los siguientes objetivos:

\begin{enumerate}
\item \textbf{Implementación del sistema de apertura de caja con \texttt{Palace\_Box\_Key}:}  
  Diseñar e integrar la lógica necesaria para permitir la apertura de la caja mediante la llave correspondiente, incluyendo detección de colisiones y validación de acceso.

\item \textbf{Mecánica de caña de pescar con efecto magnético:}  
  Desarrollar la funcionalidad de la caña con capacidad de atracción magnética, capaz de detectar objetos a distancias de 300, 60 y 5 unidades, con ajustes de precisión y respuesta visual.

\item \textbf{Animación y lógica de apertura de la caja:}  
  Crear la animación correspondiente a la apertura de la caja y programar la lógica asociada para que se ejecute correctamente tras el uso de la llave o resolución del candado.

\item \textbf{Obtención de la lámpara mágica:}  
  Integrar el objeto como recompensa al completar correctamente el puzle de la caja, incluyendo animación y lógica de recogida del objeto.
\end{enumerate}

    \item {Duración del sprint:}
El \textit{sprint } se realizó entre el 7 de mayo de 2025 y el 21 de mayo de 2025, con una duración de 14 días.

    \item {Burndown Report}
En este \textit{sprint} se cumplió con casi todos los objetivos en el plazo determinado. La implementación de modelos como la caña de pescar, la caja y los candados no habian sido proporcionados, y el atraso de anteriores \textit{sprints}, influenciaron en que este \textit{sprint} se viese atrasado ligeramente.

\item {Sprint review meeting}
Lo más destacable de esta reunión fue profundizar en la facilidad del puzle, focalizando sobre todo en la parte de los \textit{stickmans} para buscar una manera de que el usuario sepa identificar en que posición se deben de poner. Se opto por dejar la parte de la cabeza inmóvil. 
\end{itemize}
\subsection{\textbf{\textit{Sprint} 6: Palacio de Shahriar: Exportación de datos a CSV y pruebas }}
\begin{itemize}
\item {Planning meeting}:
Durante la reunión se marcaron los siguientes objetivos:

\begin{enumerate}
  \item \textbf{Exportación de métricas de rendimiento e interacción:}  
    Definir e implementar la función que registre y exporte diversas métricas del puzle —tiempos de completado de cada paso, número de intentos por paso y cualquier otro dato relevante— generando un CSV con columnas para:
    \begin{itemize}
      \item Identificación del paso
      \item Timestamp de inicio y fin
      \item Duración total
      \item Número de intentos
      %%RELLENAR CUANDO SE TERMINE
    \end{itemize}

  \item \textbf{Registro de movilidad del personaje:}  
    Diseñar un sistema de teletransporte que cambie la posición del personaje a lo largo del puzle para mejorar la accesibilidad.

  \item \textbf{Pruebas funcionales de exportación:}  
    Elaborar y ejecutar casos de prueba que verifiquen la correcta generación del CSV en distintos escenarios (puzle completado, interrumpido, reiniciado), validando integridad y consistencia de datos.
    
   \item \textbf{Añadir elementos visuales:}
   Añadir imágenes y elementos que ayuden al usuario a realizar el puzle como controles del mando y pistas.
   
\end{enumerate}

    \item {Duración del sprint:}
El \textit{sprint } se realizó entre el 21 de mayo de 2025 y el 11 de junio de 2025, con una duración de 21 días.

    \item {Burndown Report}
En este \textit{sprint} se cumplió con todos los objetivos en el plazo determinado.

    \item {Sprint review meeting}
En un principio se tenía estimado realizar otro puzle más, pero después de hablarlo y valorarlo se decidió no seguir ese plan y dedicar las últimas semanas al pulido de detalles y terminar la documentación.


\end{itemize}

\subsection{\textbf{\textit{Sprint} 7: 
Documentación final y fase de pulido}}
\begin{itemize}
\item \textit{Planning meeting}:
Durante la reunión de planificación del último \textit{sprint}, centrado en la documentación y el pulido final del proyecto, se establecieron los siguientes objetivos clave:

\begin{enumerate}
  \item \textbf{Mejora de la interfaz de usuario:}  
    Se acordó realizar ajustes visuales finales que aumenten la claridad, accesibilidad y atractivo de la experiencia, como la reorganización de elementos gráficos, tamaños de fuente o disposición de los menús en \textit{VR}.

  \item \textbf{Determinación definitiva de datos exportables:}  
    Se propuso cerrar la definición de las métricas que serán registradas en el archivo \textit{CSV}, incluyendo aquellos datos que hasta ahora estaban pendientes.

  \item \textbf{Empaquetado del proyecto:}  
    Garantizar que la versión final del sistema esté correctamente empaquetada como ejecutable (\texttt{.exe}) para \textit{Windows}, con compatibilidad completa con \textit{Meta Quest Link} y sin dependencias externas no incluidas.

  \item \textbf{Realización de pruebas funcionales y corrección de errores:}  
    Llevar a cabo una batería final de pruebas sobre las funcionalidades principales —interacción con el entorno, guardado de progreso, exportación de métricas, desplazamiento del jugador— e identificar y corregir errores remanentes.

  \item \textbf{Finalización de la documentación:}  
    Redactar y estructurar adecuadamente todos los apartados pendientes de la memoria del \textit{TFG}
\end{enumerate}

    \item {Duración del sprint:}
El \textit{sprint } se realizó entre el 11 de junio de 2025 y el 7 de julio de 2025, con una duración de 26 días.

    \item {Burndown Report}
En este \textit{sprint} se cumplió con todos los objetivos en el plazo determinado. Fue una fase larga donde se realizaron múltiple test de la aplicación, se termino de realizar la toma y exportado de datos a CSV y la correcta implementación del guardado y cargado, donde había varios \textit{bugs}.


\end{itemize}


\section{Estudio de viabilidad}
Este apartado tiene como objetivo analizar la viabilidad económica del proyecto, evaluando los costes asociados a su desarrollo y lanzamiento. Dado que el proyecto se encuentra en el ámbito académico, no tiene fines comerciales ni busca obtener beneficios económicos, pero se realiza una estimación teórica de los recursos empleados.

\subsection{Viabilidad económica}

\subsubsection{Costes}
El desarrollo del proyecto ha implicado una serie de costes que deben ser considerados para evaluar su viabilidad económica. En esta sección se detallan los diferentes componentes del coste total, incluyendo el coste de personal —tanto del desarrollador como de los tutores académicos—, el uso de hardware y software, así como otros costes indirectos derivados del uso de servicios básicos como la conexión a internet. La estimación se ha realizado teniendo en cuenta valores de mercado actuales y criterios de amortización razonables, con el fin de ofrecer una valoración lo más realista posible del esfuerzo económico asociado al desarrollo del proyecto.

\paragraph{Coste de personal}
Se estima que el desarrollo del proyecto ha requerido unas 300 horas de trabajo, distribuidas a lo largo de aproximadamente 4 meses. Esto equivale a unas 18,75 horas semanales.

Considerando que un ingeniero informático recién titulado recibe un salario bruto mensual de aproximadamente 1\,900~€ por una jornada completa de 40 horas semanales, el salario mensual equivalente para 18,75 horas semanales sería:

\[
\frac{1900~\text{€}}{40~\text{h}} \times 18{,}75~\text{h} = 890{,}63~\text{€}
\]

Este será el salario bruto del desarrollador. Para obtener el coste real para la empresa, se aplican los siguientes porcentajes según el \textit{Régimen General de la Seguridad Social}\footnote{\url{https://www.seg-social.es/wps/portal/wss/internet/CotizacionRecaudacionTrabajadores/Cotizacion/27243}}:

\begin{itemize}
  \item Contingencias comunes: 23,60\% = 0,236
  \item Mecanismo de Equidad Intergeneracional (MEI): 0,67\% = 0,0067
  \item Desempleo: 5,50\% = 0,055
  \item FOGASA: 0,20\% = 0,002
  \item Formación profesional: 0,60\% = 0,006
\end{itemize}

Aplicando estos porcentajes:

\[
890{,}63~\text{€} \times \left(1 + 0{,}236 + 0{,}0067 + 0{,}055 + 0{,}002 + 0{,}006\right) = 1\,162{,}22~\text{€}
\]

Multiplicando por los 4 meses de duración estimada del proyecto:

\[
1\,162{,}22~\text{€} \times 4 = 4\,648{,}88~\text{€}
\]

Además, se incluyen sesiones de seguimiento con los tutores del TFG. Se han realizado 8 reuniones de aproximadamente 40 minutos con 3 tutores, lo que suma un total de 16 horas de dedicación conjunta. El salario por hora de un docente con experiencia se estima en 35~€/h.

\[
16~\text{h} \times 35~\text{€/h} \times 3~\text{tutores} = 1\,680~\text{€} \text{ (bruto mensual)}
\]

Aplicando los mismos coeficientes para obtener el coste empresarial:

\[
1\,680~\text{€} \times \left(1 + 0{,}236 + 0{,}0067 + 0{,}055 + 0{,}002 + 0{,}006\right) = 2\,191{,}94~\text{€}
\]

\paragraph{Coste de hardware}
El equipo de desarrollo utilizado ha sido un ordenador de uso personal, cuyo valor estimado es de 1\,300~€. Se amortiza a 5 años (60 meses), por lo que el coste atribuido a los 4 meses de duración del proyecto es:

\[
1\,300~\text{€} \times \frac{4}{60} = 86{,}67~\text{€}
\]

El visor de realidad virtual \textit{Meta Quest 2}, también de propiedad personal, tiene un valor de mercado estimado de 350~€, amortizable igualmente a 5 años:

\[
350~\text{€} \times \frac{4}{60} = 23{,}33~\text{€}
\]

\paragraph{Coste de software}
Todas las herramientas empleadas son gratuitas para uso académico:

\begin{itemize}
  \item \textit{Unreal Engine 5}
  \item \textit{Visual Studio Community}
  \item \textit{Blender}
\end{itemize}

La única excepción es la suscripción empresarial a \textit{GitLab}, con un coste de 18~€/mes durante 4 meses:

\[
18~\text{€} \times 4 = 72~\text{€}
\]

\paragraph{Costes indirectos}
Se considera una tarifa de conexión a internet de 28~€/mes durante 4 meses:

\[
28~\text{€} \times 4 = 112~\text{€}
\]

\subsubsection*{Resumen de costes}

\begin{table}[H]
  \centering
  \begin{tabularx}{\linewidth}{@{}Xrr@{}}
    \toprule
    \textbf{Concepto} & \textbf{Duración / Cantidad} & \textbf{Coste (€)} \\
    \midrule
    \textbf{Personal} & & \\
    Desarrollador & 4 meses & 4\,648{,}88 \\
    Tutores & 16 horas & 2\,191{,}94 \\
    \textbf{Hardware} & & \\
    Ordenador de desarrollo (amortización) & 4 meses & 86{,}67 \\
    Meta Quest 2 (amortización) & 4 meses & 23{,}33 \\
    \textbf{Software} & & \\
    GitLab (suscripción empresarial) & 4 meses & 72{,}00 \\
    \textbf{Costes indirectos} & & \\
    Internet & 4 meses & 112{,}00 \\
    \midrule
    \textbf{Total estimado} & & \textbf{7\,134{,}82} \\
    \bottomrule
  \end{tabularx}
  \caption{Resumen de costes estimados del proyecto}
  \label{tab:costes-finales}
\end{table}


\subsection*{Beneficios}

El desarrollo del proyecto no persigue ningún fin comercial ni beneficio económico directo. Se trata de un proyecto académico realizado en el marco del Trabajo de Fin de Grado, orientado a la mejora de la rehabilitación motriz de personas con discapacidad en las extremidades superiores.

El objetivo principal es contribuir con una solución accesible y funcional que pueda ser utilizada, evaluada o adaptada por entidades sin ánimo de lucro como la fundación ASPAYM. La motivación del proyecto es principalmente social, centrada en aportar valor a través de la tecnología, sin que ello suponga una explotación económica del producto desarrollado.

Por tanto, los beneficios derivados del proyecto son de carácter formativo, social y tecnológico, no existiendo intenciones de monetización ni comercialización del software.


\subsection{Viabilidad legal}
La viabilidad legal de un proyecto software consiste en evaluar todos los aspectos jurídicos vinculados con el desarrollo y la distribución del producto. En este caso, se consideran:

\subsubsection{Licencias de \textit{software}}

Una licencia de \textit{software} es un acuerdo legal entre el titular de los derechos de autor y el usuario final, en el que se establecen los términos bajo los cuales se permite el uso del producto~~\cite{licenciaQueEs}.

El desarrollo del proyecto ha empleado herramientas y recursos cuya licencia es de carácter abierto o gratuito, y compatible con el uso académico. Las principales herramientas empleadas son:

\begin{itemize}
  \item \textbf{Unreal Engine 5}: Licencia de uso gratuita para fines educativos, con posibilidad de publicación sin coste mientras no se superen los umbrales de ingresos establecidos por Epic Games~~\cite{unrealLicense}.
  \item \textbf{Visual Studio Community}: Licencia gratuita para estudiantes, proyectos personales y académicos~~\cite{vslicense}.
  \item \textbf{Blender}: Licenciado bajo GNU GPLv2+, permite libre uso, distribución y modificación~~\cite{blenderlicense}.
  \item \textbf{GitLab}: Licenciado bajo MIT.
\end{itemize}

Dado que el desarrollo se ha realizado en un entorno académico, y no se contempla su distribución comercial, el cumplimiento legal respecto a las licencias se mantiene dentro de los márgenes establecidos por los proveedores de las herramientas utilizadas.

\subsubsection{Licencias de imágenes y contenido gráfico}

Para el diseño de interfaces o materiales visuales incluidos en el entorno de realidad virtual, se han empleado únicamente recursos libres de derechos o generados específicamente para este proyecto. 

Todas las fuentes de iconos o modelos utilizados se han documentado debidamente en el repositorio del proyecto, siguiendo los términos de uso de cada proveedor.

\subsubsection{Licencia del proyecto desarrollado}

El software desarrollado será distribuido bajo una licencia libre. En este caso, se ha optado por aplicar la licencia \textbf{GNU GPLv3}, ya que garantiza que cualquier adaptación o distribución futura del software mantenga las mismas condiciones de libertad y apertura que el original. Esta decisión se alinea con el carácter académico, social y sin ánimo de lucro del proyecto.

\subsubsection{Licencia de la documentación}

La documentación generada para el proyecto será publicada bajo la licencia \textbf{Creative Commons CC BY-NC-ND 4.0}, lo que permite su libre distribución siempre que se mencione la autoría, no se utilice con fines comerciales y no se generen obras derivadas ~\cite{cclicenses}.

\subsubsection{Cumplimiento de normativa de protección de datos}

El proyecto no trata datos personales reales, ya que la aplicación está diseñada para no recoger ningún tipo de dato personal, tan solo datos sobre el progreso realizado en los puzles. Por tanto, no aplica el RGPD ni la LOPDGDD en este contexto.

\subsection{Licencia de distribución de software}

Tras el análisis de compatibilidad de licencias con las herramientas utilizadas en el desarrollo, se ha optado por distribuir el código fuente del proyecto bajo la licencia \textbf{MIT License}. Esta elección permite una reutilización amplia y flexible del código, incluso en proyectos comerciales o cerrados, siempre que se mantenga la atribución al autor original. 

La \textit{MIT License} resulta especialmente adecuada para proyectos académicos y de investigación, ya que fomenta la difusión abierta del conocimiento sin imponer restricciones adicionales. Asimismo, es compatible con el uso del motor \textit{Unreal Engine}, cuyo entorno de desarrollo no impide la aplicación de licencias permisivas en el código propio del desarrollador.

Esta decisión está alineada con el objetivo de facilitar la adopción, mejora y extensión del sistema por parte de terceros, incluyendo comunidades educativas, clínicas y entidades colaboradoras como ASPAYM. Se incluye el texto completo de la licencia en el repositorio del proyecto, junto con la atribución correspondiente en los archivos fuente principales.

\begin{table}[p]
	\centering
	\begin{tabularx}{\linewidth}{ p{0.50\columnwidth} p{0.50\columnwidth} }
		\toprule
		\textbf{Recurso} & \textbf{Licencia} \\
		\toprule
		\textbf{Modelos ASPAYM} & 
			CC BY-NC-SA 4.0 \\
		\addlinespace
		\textbf{Código propio (GitLab)} &
			MIT License \\
		\addlinespace
		\textbf{Unreal Engine} &
			EULA \\
		\addlinespace
		\textbf{Blender} &
			GNU GPL v2 o superior \\
		\bottomrule
	\end{tabularx}
	\caption{Licencias de dependencias y recursos del proyecto}
	\label{tab:licencias-dependencias}
\end{table}




