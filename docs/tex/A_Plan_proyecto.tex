\apendice{Plan de Proyecto Software}

\section{Introducción}

\section{Planificación temporal}
La planificación temporal de este proyecto, debido a que se esta siguiendo la metodologia Scrum, sera realizada mediante sprints.

\subsection{\textbf{Sprint 1: Planificación del proyecto y estructura del juego}}
\begin{itemize}
\item {Planning meeting}
Durante la reunión se marcaron los siguientes objetivos:

\begin{enumerate}
\item \textbf{Revisión de documentación de puzzles:}
  Leer y analizar toda la documentación relacionada con los puzzles para comprender los requisitos,  
  mecánicas y desafíos que deberán implementarse.

\item \textbf{Planificación de la estructura del juego:}
  Definir la arquitectura general del juego, estableciendo los módulos  
  principales, el flujo de navegación y la integración de los puzzles.

\item \textbf{Selección de gafas de realidad virtual:}
  Acordar la versión de las gafas de \textit{RV} a utilizar y definir la  
  preferencia basada en la compatibilidad, rendimiento y experiencia de usuario.
\end{enumerate}

    \item {Duración del sprint}
El \textit{sprint } se realizo entre el 12 de marzo de 2025 y el 17 de marzo de 2025, con una duración de 7 días.

    \item {Burndown Report}
Este sprint se cumplio con casi todos los objetivos en el plazo determinado. La selección de las gafas de realidad virtual se veia retrasada debido a que la persona que iba a prestarlas para este proyecto no se encontraba disponible es ese momento. Cabe destacar que durante la duración de este sprint tambien se realizaron tareas de aprendizaje sobre Unreal Engine ya que los conocimientos eran nulos sobre este motor grafico. 

    \item {Sprint review meeting}
Durante esta reunión, se resolvieron  dudas sobre el sistema de registro que debia tener la aplicación, como se querian estructurar las partidas y el requisito de toma de algunos datos que pedia la organización ASPAYM que eran inviables en el nivel de este proyecto. Los issues fueron creados de forma correcta, pero se marco que habia que planificar los sprints correspondientes al proyecto.
\end{itemize}
\subsection{\textbf{Sprint 2: Menu principal } }
\begin{itemize}
    \item {Planning meeting}
    
    Durante la reunión se marcaron los siguientes objetivos:
    \begin{enumerate}
        \item \textbf{Registro de usuarios de forma local:}El registro se realizará mediante el \textit{savegame} de la partida, donde, si es la primera vez que se inicia el juego en el dispositivo, solo se pedirá al usuario el \textbf{nombre}. De esta manera, se asegura que el registro sea rápido y sencillo. 
    
        \item \textbf{Creación de nuevas partidas:}Cada partida será asociada a un nivel, el cual se seleccionará como el puzzle específico que se quiere jugar. Esto implica que cada partida estará vinculada a un puzzle que servirá como contenido principal de la misma. 
    
        \item \textbf{Listado de las partidas creadas:} La aplicación deberá permitir visualizar una lista de las partidas y proporcionar información básica sobre cada puzzle, como el nombre del puzzle, el nivel y el estado del progreso (si corresponde). 
    \end{enumerate}
    
    
        \item Duración del sprint
    
    El \textit{sprint } se realizo entre el 26 de marzo de 2025 y el 9 de abril de 2025, con una duración de 14 días.
    
    
        \item {Burndown Report}
    
    Este sprint se cumplio con los objetivos en el plazo determinado. La estimación de tiempo fue muy extensa debido a que se utilizo este plazo para seguir adquiriendo conocimientos sobre \textit{Unreal Engine }de mientras se realizaban los objetivos del sprint.
    
    
        \item {Sprint review meeting}
    
    Se hablo sobre la importancia de agrupar los elementos de la interfaz en \textit{plantillas} como los botones e \textit{items} para la reusabilidad, ya que no tuve en cuenta algunos elementos por lo cual ese aspecto se planteo cambiar para el siguiente sprint.
\end{itemize}

\subsection{\textbf{Sprint 3: Plantilla de puzzle e integración de menu principal a VR } }
\begin{itemize}
    \item {Planning meeting}
    
Durante la reunión se marcaron los siguientes objetivos:
\begin{enumerate}
  \item \textbf{Integración de interfaces a realidad virtual:}  
    Adaptar y conectar todas las pantallas y controles del juego al entorno RV, garantizando la navegación y la interacción fluida.

  \item \textbf{Contador de tiempo y almacenamiento de nivel:}  
    Implementar un temporizador que comience al iniciar el puzzle y guarde el tiempo transcurrido junto con los datos del nivel creado en la sesión.

  \item \textbf{Cargado de una partida:}  
    Desarrollar la funcionalidad para seleccionar y restaurar el estado de una partida previamente guardada, incluyendo el progreso y el nivel activo.

  \item \textbf{Visualización de resultados del puzzle:}  
    Mostrar al finalizar cada puzzle los siguientes datos:  
    \begin{itemize}
      \item Tiempo empleado  
      \item Fecha de creación de la sesión  
      \item Nombre del puzzle jugado
      \item Completado o no completado
    \end{itemize}
\end{enumerate}
    \item {Duración del sprint}
    
El \textit{sprint } se realizo entre el 9 de abril de 2025 y el 23 de abril de 2025, con una duración de 14 días.
    \item {Burndown Report}
    
Este sprint no se cumplio con los objetivos en el plazo determinado. La estimación de tiempo puede que fuera la correcta pero la falta del hardware retrasa las pruebas del software debido a no tener acceso a unas gafas de realidad virtual por la festividad de semana santa.Se focalizo más en la parte de documentación del proyecto.

    \item {Sprint review meeting}

\end{itemize}
\section{Estudio de viabilidad}

\subsection{Viabilidad económica}

\subsection{Viabilidad legal}


