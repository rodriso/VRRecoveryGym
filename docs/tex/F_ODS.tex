\apendice{Anexo de sostenibilización curricular}

\section{Introducción}
La sostenibilidad es un eje fundamental para afrontar los retos sociales, ambientales y económicos del presente y del futuro. Su integración en el contexto universitario permite al estudiantado desarrollar competencias clave para promover un desarrollo más justo, eficiente y respetuoso con el entorno.

El Trabajo de Fin de Grado desarrollado, contribuye a varios de los Objetivos de Desarrollo Sostenible (ODS) establecidos por la ONU en la Agenda 2030. Esta aplicación de realidad virtual no solo busca mejorar la rehabilitación motriz de personas con discapacidad en extremidades superiores, sino que lo hace desde una perspectiva inclusiva, accesible y digitalmente eficiente.

A continuación, se exponen los ODS más relevantes relacionados con el proyecto.

\section{ODS 3: Salud y Bienestar}
El objetivo principal de \textit{VRRecoveryGym} es facilitar la rehabilitación de usuarios con dificultades de movilidad, contribuyendo directamente a su bienestar físico y emocional. El sistema permite realizar ejercicios terapéuticos a través de puzles interactivos en un entorno virtual accesible desde el propio domicilio.

\subsection*{Impacto en la práctica:}
\begin{itemize}
  \item \textbf{Rehabilitación accesible desde casa:} Los usuarios pueden realizar sus sesiones sin necesidad de desplazarse a las instalaciones físicas de la fundación.
  \item \textbf{Bienestar emocional:} El hecho de poder elegir cuándo y dónde realizar los ejercicios reduce la carga psicológica asociada al tratamiento.
  \item \textbf{Estímulo cognitivo y motor:} El entorno gamificado incentiva la participación y la repetición terapéutica.
\end{itemize}

\section{ODS 9: Industria, Innovación e Infraestructura}
El proyecto incorpora tecnologías emergentes como la realidad virtual y el desarrollo en Unreal Engine para proponer una solución innovadora a un problema tradicional del ámbito sanitario. Esto permite democratizar el acceso a recursos terapéuticos de calidad.

\subsection*{Impacto en la práctica:}
\begin{itemize}
  \item \textbf{Digitalización del tratamiento:} Eliminación del soporte en papel para el seguimiento y registro de datos, que ahora se almacenan digitalmente mediante \textit{SaveGames}.
  \item \textbf{Uso eficiente de recursos:} No se requiere infraestructura física adicional, ya que se ejecuta en un entorno doméstico con equipamiento común.
\end{itemize}

\section{ODS 10: Reducción de las desigualdades}
Al permitir que cualquier persona con un visor VR y un ordenador compatible pueda acceder al tratamiento, se reducen barreras geográficas y económicas. Esto facilita la inclusión de personas con movilidad reducida o con acceso limitado a centros especializados.

\subsection*{Impacto en la práctica:}
\begin{itemize}
  \item \textbf{Accesibilidad:} Solución adaptable a diferentes perfiles de usuario sin necesidad de asistencia externa.
  \item \textbf{Independencia:} Empodera al usuario en su propio proceso terapéutico.
\end{itemize}

\section{ODS 13: Acción por el clima}
Aunque de forma indirecta,el proyecto también contribuye al ODS 13 al reducir la huella ambiental asociada a los desplazamientos físicos y al uso de materiales impresos. Cada sesión realizada desde casa implica una menor emisión de gases de efecto invernadero, al evitar el uso de transporte privado o colectivo hacia instalaciones especializadas.

Además, al prescindir completamente del papel en el registro y seguimiento de la actividad, el sistema evita el consumo de recursos naturales y la energía empleada en los procesos de impresión, archivo y distribución.

\subsection*{Impacto en la práctica:}
\begin{itemize}
  \item \textbf{Reducción de desplazamientos:} El usuario no necesita acudir presencialmente a la fundación, lo que implica una menor dependencia de transporte motorizado.
  \item \textbf{Disminución del uso de papel:} Toda la información relativa al progreso del jugador y al estado del entorno se guarda digitalmente, eliminando la necesidad de formatos físicos.
  \item \textbf{Sostenibilidad digital:} El modelo propuesto apuesta por tecnologías limpias para gestionar la rehabilitación sin generar residuos materiales.
\end{itemize}

\section{Conclusión}
\textit{VRRecoveryGym} representa una aproximación tecnológica al servicio de la sostenibilidad humana y social. Permite transformar un proceso tradicionalmente presencial y costoso en una experiencia accesible, digital y centrada en el bienestar del usuario.
