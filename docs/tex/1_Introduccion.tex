\capitulo{1}{Introducción}


Vivimos en una época donde la tecnología se ha convertido en una herramienta clave para mejorar la calidad de vida de las personas. En particular, el desarrollo de soluciones tecnológicas aplicadas al ámbito de la salud ha abierto nuevas posibilidades para abordar desafíos de rehabilitación física mediante enfoques innovadores. En este contexto, la realidad virtual (VR) ha demostrado un gran potencial para motivar y asistir a pacientes en procesos de recuperación motriz, especialmente en colectivos con discapacidades físicas que requieren terapias continuadas y personalizadas.

Actualmente, se utilizan puzles físicos impresos en 3D para promover movimientos específicos que permiten la rehabilitación de estas personas. Sin embargo, este enfoque presencial limita la autonomía de los usuarios y requiere supervisión constante por parte de profesionales.

El presente proyecto busca trasladar estos ejercicios a un entorno virtual accesible desde el domicilio del usuario. Mediante el uso de cascos de realidad virtual y controladores (o \textit{hand tracking} en un futuro), se desarrollarán actividades interactivas que simulan los puzles físicos. Además, se recopilarán métricas clave sobre la actividad motriz y el progreso del usuario, que podrán ser consultadas por profesionales de forma remota mediante un \textit{CSV} exportado con todos los datos relevantes sobre el progreso del paciente en los puzles.

El desarrollo se llevará a cabo utilizando Unreal Engine, combinando programación en C++ y \textit{Blueprint}, así como recursos 3D entregados, ya realizados por la fundación, y el \textit{marketplace} de Unreal. El proyecto se beneficia del asesoramiento de los profesionales de ASPAYM para garantizar la validez terapéutica del sistema.
