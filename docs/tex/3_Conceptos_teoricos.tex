\capitulo{3}{Conceptos teóricos}
En esta sección se presentan los fundamentos teóricos necesarios para comprender el desarrollo y la implementación del presente Trabajo de Fin de Grado. Se abordan las principales definiciones, teorías y modelos que sustentan la investigación, así como las tecnologías y herramientas empleadas. 
\section{Realidad virtual}
Se entiende por realidad virtual (RV) la tecnología que permite crear entornos digitales tridimensionales, interactivos y en tiempo real, de forma que el usuario perciba la sensación de “estar presente” en dicho entorno. Este grado de inmersión y “telepresencia” se consigue mediante la combinación de hardware (cascos HMD, sistemas de tracking, interfaces hápticas) y software (motores gráficos, algoritmos de simulación) que actualizan la escena virtual en función de los movimientos y acciones del usuario \cite{steuer92,sherman2002}.

Dentro del estudio de la RV suelen distinguirse tres características fundamentales:

\textbf{Inmersión:} grado en que el sistema aísla y sustituye estímulos sensoriales reales por estímulos generados artificialmente, creando la impresión directa de “estar dentro” del entorno virtual \cite{milgram94}.

\textbf{Presencia:} sensación subjetiva de “estar realmente” en el escenario virtual, más allá de la mera percepción sensorial \cite{sherman2002,slater94}.

\textbf{Interactividad:} capacidad del sistema para registrar las acciones del usuario y modificar el entorno virtual de forma coherente y en tiempo real \cite{steuer92,milgram94}.

Estas características hacen de la RV una herramienta poderosa en ámbitos tan diversos como la simulación de entrenamiento, la rehabilitación médica, la visualización arquitectónica o la investigación científica.

\subsection{Componentes de un sistema de RV}

Un sistema de RV típico se compone de \cite{sherman2002,burdea2003}:
\begin{itemize}
  \item \textbf{Dispositivo de visualización (Head-Mounted Display, HMD):} muestra las imágenes estereoscópicas e incorpora sensores de orientación (giroscopios, acelerómetros).
  \item \textbf{Sistema de seguimiento (tracking):} rastrea la posición y orientación de la cabeza, manos u otros objetos mediante cámaras, emisores infrarrojos o sistemas magnéticos.
  \item \textbf{Dispositivos de interacción:} mandos, guantes hápticos o superficies táctiles que permiten manipular objetos virtuales.
  \item \textbf{Motor gráfico y software de simulación:} genera las escenas 3D, gestiona física, colisiones y comportamientos de los elementos del entorno.
\end{itemize}

\subsection{Taxonomía: Continuo de la virtualidad}

Milgram y Kishino propusieron el concepto de continuo de la virtualidad, que va desde la realidad completamente física hasta la realidad completamente virtual, englobando la realidad aumentada (AR) y la realidad mixta asistida \cite{milgram94}:
\begin{itemize}
  \item \textbf{Realidad Aumentada (AR):} superposición de información virtual sobre el entorno real.
  \item \textbf{Realidad Mixta (MR):} fusión de elementos reales y virtuales que interactúan entre sí.
  \item \textbf{Realidad Virtual (RV):} reemplazo total del entorno real por uno virtual.
\end{itemize}

\subsection{Modalidades de uso y aplicaciones}

Según el grado de inmersión y el tipo de interacción, los sistemas de RV pueden clasificarse en:
\begin{itemize}
  \item \textbf{RV de escritorio:} utiliza pantallas convencionales y controladores estándar; menor inmersión, pero accesible.
  \item \textbf{RV inmersiva:} emplea HMDs y sistemas de tracking avanzado; alta inmersión y presencia \cite{burdea2003}.
  \item \textbf{CAVEs:} espacios cerrados con pantallas proyectadas en paredes y suelo; múltiples usuarios simultáneos.
\end{itemize}

Las aplicaciones más difundidas incluyen simuladores de vuelo y conducción, terapias de exposición en salud mental, entrenamiento quirúrgico y entornos de aprendizaje inmersivo.

\section{Diseño de videojuegos para rehabilitación}

El uso de videojuegos para la rehabilitación física se enmarca dentro del concepto de \textit{serious games}, definido como aquellos juegos cuyo principal objetivo va más allá del entretenimiento, buscando también la educación, la formación o la mejora de la salud \cite{david2016serious}.

\subsection{Principios de diseño}

El desarrollo de videojuegos para rehabilitación debe considerar aspectos tanto lúdicos como terapéuticos \cite{greene2016rehab}. Entre los principales principios de diseño se encuentran:

\begin{itemize}
    \item \textbf{Interactividad:} permite al usuario manipular y explorar el entorno, facilitando el aprendizaje y la mejora de habilidades motoras.
    \item \textbf{\textit{Feedback} inmediato:} el sistema debe ofrecer respuestas instantáneas a las acciones del usuario, reforzando el progreso \cite{cameirao2010impact}.
    \item \textbf{Dificultad progresiva:} las tareas deben adaptarse al nivel de capacidad del paciente, aumentando la complejidad de forma gradual para evitar la frustración \cite{zimmerli2012virtual}.
    \item \textbf{Repetición y refuerzo positivo:} elementos clave en la rehabilitación física para promover la neuroplasticidad.\cite{holden2005virtual}.
\end{itemize}

\subsection{Gamificación y motivación}

La gamificación introduce mecánicas de juego (puntos, logros, niveles) que aumentan la motivación y el compromiso del usuario \cite{sardi2017gamification}. En el caso del presente proyecto, la superación de puzzles y la obtención de logros sirven como incentivo para continuar con las sesiones terapéuticas.

\subsection{Accesibilidad y consideraciones éticas}

Los videojuegos de rehabilitación deben garantizar la accesibilidad a personas con diferentes grados de discapacidad. Además, se debe asegurar que el uso de la tecnología no sustituya la supervisión profesional, sino que actúe como complemento \cite{rodriguez2020reha}.
