\capitulo{2}{Objetivos del proyecto}

Este trabajo se enmarca en el ámbito de la rehabilitación física asistida por tecnologías inmersivas. Se propone el desarrollo de una herramienta innovadora en forma de entorno de realidad virtual para favorecer la recuperación motora de personas con lesión medular o discapacidad física en las extremidades superiores mediante la realización de puzles.

\section{Objetivos}
Los objetivos se dividen en dos categorías: técnicos y de desarrollo software.

\subsection{Objetivos técnicos}
\label{Objetivos técnicos}
\begin{enumerate}
    \item \textbf{Simular puzles físicos en un entorno de realidad virtual:} reproducir los ejercicios físicos diseñados por ASPAYM mediante interacción virtual precisa.
    
    \item \textbf{Registrar métricas motrices relevantes:} como número de repeticiones, rangos de movimiento, tiempos de resolución o errores durante la actividad.
    
    \item \textbf{Facilitar la evaluación del progreso del usuario:} mediante informes visuales y datos exportables para su análisis por parte de profesionales.
    
    \item \textbf{Permitir el uso remoto de la herramienta:} asegurando accesibilidad desde entornos domésticos sin supervisión directa constante.
    
    \item \textbf{Explorar la viabilidad del \textit{hand tracking}:} como alternativa a los mandos que acompañan las gafas de realidad virtual, buscando una experiencia más natural y accesible.
    
    \item \textbf{Garantizar la estabilidad y rendimiento del entorno virtual:} con pruebas de usabilidad y detección temprana de errores.
\end{enumerate}

\subsection{Objetivos de desarrollo software}
\label{Objetivos de desarrollo}
\begin{enumerate}
    \item \textbf{Desarrollar el entorno de realidad virtual utilizando Unreal Engine:} combinando C++ y \textit{Blueprints} para lograr un equilibrio entre rendimiento y modularidad.
    
    \item \textbf{Implementar y optimizar modelos 3D con Blender:} representando fielmente los puzles físicos y el entorno de trabajo.
    
    \item \textbf{Seguir un marco ágil de desarrollo (SCRUM):} organizando el trabajo en \textit{sprints} y manteniendo una comunicación clara mediante GitLab.
    
    \item \textbf{Realizar pruebas funcionales:} que permitan verificar la funcionalidad del sistema en distintas etapas de desarrollo.
    
    \item \textbf{Preparar el sistema para su despliegue en dispositivos de realidad virtual:} inicialmente en una versión específica de dispositivo, con posibilidad de extenderlo a más dispositivos.
\end{enumerate}