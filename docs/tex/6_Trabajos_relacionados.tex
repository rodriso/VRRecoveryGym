\capitulo{6}{Trabajos relacionados}

Este proyecto se enmarca en la intersección de la realidad virtual, los videojuegos y la rehabilitación motora. Se han identificado diversas líneas de investigación relevantes, que se detallan a continuación.

\subsection{Realidad virtual inmersiva para rehabilitación motora}

La realidad virtual inmersiva ha demostrado ser efectiva en la rehabilitación de extremidades superiores, especialmente en pacientes que han sufrido accidentes cerebrovasculares. Estudios recientes como en el articulo \textit{''Virtual Reality Therapy for Upper Limb Motor Impairments in Stroke Patients: A Meta-Analysis'' }~\cite{vr_meta_stroke2024}, indican que la combinación de RV con terapias tradicionales mejora la función motora y la destreza manual, siendo más efectiva durante las etapas aguda y subaguda de la recuperación.

Además, la VR permite a los pacientes realizar tareas funcionales repetitivas de manera lúdica, aumentando la adherencia al tratamiento y la motivación ~\cite{vr_upper_extremity2023}.

\subsection{Desarrollo de videojuegos terapéuticos con Unity y Unreal Engine}

En esta sección, es relevante citar el trabajo \textit{''Design and usability evaluation of an immersive virtual reality mirrored hand system for upper limb stroke rehabilitation'' }~\cite{mirrored_hand_system2025}. Se ha desarrollado un sistema de realidad virtual inmersiva con actividades de la vida real, como cocinar o conducir, para la rehabilitación del miembro superior en las personas con accidentes cerebrovasculares.

Asimismo, proponen un tipo de videojuego interactivo que combina el juego con la actividad física como terapia virtual que utiliza sensores portátiles inteligentes para la rehabilitación de la extremidad superior, capturando movimientos de la mano y el codo para personalizar la terapia, recogido en el siguiente trabajo ~\cite{exergame_sensors2023}.

\subsection{Gamificación y motivación en la terapia VR}

La gamificación en la VR ha mostrado beneficios en la rehabilitación, como la mejora de la función motora y la reducción del dolor percibido. Por ejemplo, en el artículo\textit{''Playing your pain away: designing a virtual reality physical therapy for children with upper limb motor impairmen''}      ~\cite{vr_kids_pain2023},se ha diseñado una terapia física de realidad virtual para niños con discapacidad motora en las extremidades superiores, mejorando la duración del ejercicio y produciendo emociones positivas hacia la terapia.

\subsection{Integración de VR y robótica en la rehabilitación}

La combinación de VR con dispositivos robóticos ha sido explorada para mejorar la rehabilitación de la extremidad superior. El estudio \textit{``Immersive Virtual Reality and Robotics for Upper Extremity Rehabilitatio''}~\cite{vr_robotics2023}, introdujo una solución de rehabilitación virtual que combina VR con robótica y sensores portátiles para analizar los movimientos de la articulación del codo, mostrando ventajas potenciales en un enfoque inmersivo y multisensorial.