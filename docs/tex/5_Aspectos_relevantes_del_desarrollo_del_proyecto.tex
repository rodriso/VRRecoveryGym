\capitulo{5}{Aspectos relevantes del desarrollo del proyecto}

Uno de los aspectos más significativos de este proyecto ha sido la oportunidad de trabajar en un entorno real con un cliente externo. En este caso, el cliente fue el responsable de la asociación ASPAYM, quien participó activamente en las reuniones de seguimiento del proyecto. Su implicación permitió establecer una comunicación directa y constante, aportando ideas, validando propuestas y sugiriendo modificaciones en función de las necesidades reales del colectivo al que va dirigido el sistema. Esta experiencia ha sido especialmente enriquecedora, ya que ha permitido integrar la perspectiva del usuario final en todas las fases del desarrollo, adaptando el enfoque inicial a una solución verdaderamente útil y accesible.

Otro punto destacable ha sido el proceso de aprendizaje en torno al desarrollo de aplicaciones en \textit{realidad virtual}, un campo completamente nuevo al inicio del trabajo. El uso de \textit{Unreal Engine}, junto con el desarrollo para dispositivos \textit{Meta Quest}, ha supuesto un reto técnico importante que ha requerido una formación autodidacta intensiva. A lo largo del proyecto, se ha adquirido experiencia en el diseño de entornos virtuales inmersivos, implementación de mecánicas interactivas y gestión de la lógica del juego en un entorno 3D, aplicando tanto programación en \textit{Blueprints} como nociones básicas de \textit{C++}.

Además, el uso de recursos 3D ha sido fundamental para el diseño de los puzles interactivos del sistema. Se han integrado modelos tridimensionales que no solo cumplen una función estética si no que ha requerido un trabajo minucioso de adaptación, configuración de colisiones y ajuste de físicas para garantizar una experiencia fluida, funcional y segura para el usuario.

Por ultimo, resulta crucial haber identificado desde el inicio las particularidades de los usuarios finales, muchos de los cuales presentan limitaciones motrices derivadas de su condición. Diseñar la jugabilidad pensando en ellas implicó adoptar mecánicas que reduzcan la necesidad de movimientos precisos y de alta velocidad, así como ofrecer múltiples modalidades de interacción—ya sea mediante gestos amplios y simplificados, menús accesibles o ayudas visuales y sonoras que guíen al jugador. Este enfoque centrado en la accesibilidad no solo mejora la experiencia de quienes tienen dificultades físicas, sino que también enriquece el sistema general al fomentar una interfaz intuitiva y adaptable, capaz de ajustarse a distintos niveles de destreza y garantizar una participación inclusiva y satisfactoria.