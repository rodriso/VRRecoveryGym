\capitulo{4}{Técnicas y herramientas}

En esta sección se presentan algunas de las herramientas fundamentales empleadas en el desarrollo del proyecto.
\section{Herramientas}
\subsection{Unreal Engine}

Unreal Engine es un motor gráfico y de desarrollo de juegos en tiempo real desarrollado por Epic Games. Su arquitectura modular y su potente editor visual permiten diseñar, iterar y desplegar entornos 3D de alta fidelidad para aplicaciones que van desde videojuegos hasta simulaciones VR y AR. Para este proyecto se ha utilizado Unreal Engine 5, aprovechando sus siguientes características principales:

\begin{itemize}
  \item \textbf{Editor de niveles y blueprints:} interfaz WYSIWYG para construir escenas y lógica de juego mediante nodos, sin necesidad de programar directamente en C++.
  \item \textbf{Renderizado de última generación:} soporte para Nanite (geometría virtualizada) y Lumen (iluminación global dinámica), que permiten obtener gráficos realistas en tiempo real.
  \item \textbf{Soporte nativo de Realidad Virtual:} integración con SDKs de dispositivos HMD (Oculus, SteamVR, etc.), sistemas de tracking y controladores hápticos.
\end{itemize}

Para la integración de la interfaz de usuario en RV se ha empleado el sistema UMG (Unreal Motion Graphics), lo que ha permitido diseñar menús y HUDs directamente dentro del entorno 3D.

\subsection{Blueprints} 
Las Blueprints constituyen el sistema de scripting visual de Unreal Engine que permite definir la lógica de juego y comportamiento de objetos sin escribir código en C++ directamente. Basado en un paradigma de programación orientada a objetos y data flow, Blueprints ofrece una interfaz de nodos conectables que representan:

\textbf{Clases y Actores:} cada Blueprint es esencialmente una subclase de \textit{UObject} o \textit{AActor}, con variables (propiedades), funciones y eventos propios.

\textbf{Graph Editor:} el Event Graph agrupa nodos de eventos (p. ej., BeginPlay, Tick, entradas de usuario) con nodos de ejecución (funciones, macros, secuencias lógicas) que conectan flujos de datos y control.

\textbf{Construction Script:} un grafo especial que se ejecuta en el editor para inicializar propiedades y componentes antes del runtime.

\textbf{Typed Nodes:} Blueprints maneja múltiples tipos de nodos (“Event”, “Function”, “Macro”, “Interface”, “Custom Event”), cada uno con firma de entrada/salida, lo que facilita la reutilización de lógica.

\textbf{Compilación y VM:} al guardar un Blueprint, Unreal compila el grafo a un bytecode que corre en la Blueprint Virtual Machine, garantizando rendimiento y depuración (breakpoints, watch variables) similares a código nativo \cite{epicBlueprints}.

\subsection{Blender}

Blender es una suite de creación 3D de código abierto que incluye herramientas para modelado, escultura, texturizado y exportación de geometrías. En este proyecto, su uso se ha centrado exclusivamente en la preparación de los modelos 3D proporcionados en formato STL y su posterior exportación para Unreal Engine:

\begin{itemize}
  \item \textbf{Importación de STL:} carga de los ficheros STL originales, comprobación de integridad del modelo.
  \item \textbf{Ajuste de escala y orientación:} normalización de las dimensiones del modelo y alineación de los ejes para que coincidan con las unidades y el sistema de coordenadas de Unreal Engine.
  \item \textbf{Conversión y exportación:} exportación del modelo en formatos compatibles (FBX u OBJ), asegurando que las normales de las caras permanezcan orientadas correctamente y que no se pierda información geométrica al importar en Unreal Engine.
  
\end{itemize}